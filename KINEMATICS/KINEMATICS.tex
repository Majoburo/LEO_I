%                                                                 aa.dem
% AA vers. 8.2, LaTeX class for Astronomy & Astrophysics
% demonstration file
%                                                       (c) EDP Sciences
%-----------------------------------------------------------------------
%
%\documentclass[referee]{aa} % for a referee version
%\documentclass[onecolumn]{aa} % for a paper on 1 column  
%\documentclass[longauth]{aa} % for the long lists of affiliations 
%\documentclass[rnote]{aa} % for the research notes
%\documentclass[letter]{aa} % for the letters 
%\documentclass[bibyear]{aa} % if the references are not structured 
% according to the author-year natbib style

%
%\documentclass[twocolumn]{aastex61}

\documentclass[%
 aip,
 twocolumn,
 jmp,%
 amsmath,amssymb,
%preprint,%
 reprint,%
%author-year,%
%author-numerical,%
]{aastex61}
\usepackage{amsmath}
\usepackage{graphicx}% Include figure files
\usepackage{dcolumn}% Align table columns on decimal point
\usepackage{bm}% bold math
\usepackage{relsize}
\usepackage{subfiles}

%\usepackage{aastex}

%\usepackage[mathlines]{lineno}% Enable numbering of text and display math
%\linenumbers\relax % Commence numbering lines
\newcommand{\Msun}{$M_{\odot}$}
\begin{document}

%\preprint{AIP/123-QED}

%\title[MEASURING DARK MATTER PROFILES NON-PARAMETRICALLY IN DWARF SPHEROIDALS]{MEASURING DARK MATTER PROFILES NON-PARAMETRICALLY IN DWARF SPHEROIDALS:
%	AN APPLICATION TO LEO I}% Force line breaks with \\
%\thanks{Footnote to title of article.}


%\date{\today}% It is always \today, today,
             %  but any date may be explicitly specified



%\pacs{Valid PACS appear here}% PACS, the Physics and Astronomy
                             % Classification Scheme.
%\keywords{Suggested keywords}%Use showkeys class option if keyword
                              %display desired



%\section{\label{sec:level1}}

\subsection{KINEMATICS}


In terms of the kinematics, we dealt with two different datasets:
\begin{itemize}
    \item Individual fiber spectra from VW.
    \item Individual fiber velocities and errors from M98.
\end{itemize}

As discussed in section 3.4, a fiber, when placed close to LEO I's center, receives  light from many stars at a time. This biases the measured fiber velocity towards the galaxy's velocity and at the same time broadens the spectral absorption features.

In the large sample limit, this will just be another example of the central limit theorem (the regular principle by which one measures galactic velocity dispersions, a.k.a. integrated light measurement).

At the small sample limit though, the problem becomes analytically intractable when only supplied with the measured fiber velocity and corresponding magnitudes of the contributing stars. Even though observationally one could imagine performing spectral decomposition of the final spectra, this would require superb signal to noise of both templates and spectra, with is not our case.

Hence, one is faced with two limiting cases on which standard techniques could be applied to recover the dispersion; the many or the one star in fiber case. The in between is generally unprobed territory.

As discussed in detail in section 3.4, this paper attempts to address the small numbers limit introducing a simple yet novel way to estimate the dispersion in the small numbers limit.

In this section we will focus on the techniques dealing with both the many and one star in fiber case, i.e. integrated light measurements and cross correlations. The later will then be further treated for corrections due to unresolved crowding.

\subsubsection{INTEGRATED LIGHT MEASUREMENTS}

There is probably order of magnitude as many kinematic codes measuring integrated light as observers in our field. The principle behind their operation is the following: 
\begin{itemize}
\item A linear combination of stellar templates is convolved with a set of functions (parametric or non-parametric) and the result that minimizes the Chi2 fit with the measured spectra is selected.
\item The function responsible for such best fitting spectra is taken to be the representative of the probability distribution of velocities within the given spatial region from which light was integrated.
\end{itemize}
In this paper we used Karl Gebhardt's non-parametric code \texttt{npdyne} [[Should we also talk about/overlay the PPXF results??]] to measure integrated light. As input for this calculation we used a library of stellar templates priorly extracted with VW using the procedure described in the above subsection. We further selected those templates to match the stellar population we saw best described in the color magnitude diagram performed using the HST imaging referred in section ??.

As seen in figure ??, most fibers within the VW spatial coverage contained ~40 stars and had about ~40\% of the light coming from their brightest contributing star. This meant in order to reach the large numbers limit (~3\% contribution per data point) we required ~12 fibers. [[CAUTION: BALLPARK ESTIMATION PARAGRAPH, I COULD USE THE Lindeberg–Lévy CLT MORE FORMALLY IF NEED BE (BUT TRULY WE DID IT JUST BY EYEBALL)]]

Following this logic, we stacked ~12 fibers per $(r,\theta)$ bin (Figure ??) and proceeded to measure their dispersion. Non only did this help in reaching the large sample limit but also increased our signal to noise significantly.

Figure ?? Spectra of our different $(r,\theta)$ bins.

Figure ?? Color coded bins of Leo I.

Figure ?? LOSVDs for different bins

\subsubsection{NORMALIZED CROSS-CORRELATION}

In addition to this, we performed normalized cross-correlations on each fiber spectra. This was done basically to highlight the effect of unresolved crowding in the small sample limit. Figure ?? shows the dispersions for each calculation. Talk here more about how cross-correlations work???

\subsubsection{M98 AND UNRESOLVED CROWDING CORRECTIONS}

Finally, fiber velocities taken in M98 where rebinned in concentric annuli matching isocontours of the luminosity density profile. A subset of such fibers fell within deep HST imaging which enabled us to check them for unresolved crowding. 
The final LOSVD profile was then finally made taking both the our measurements for the inner bins and M98 measurements corrected for unresolved crowding for the outer ones. Figure ?? shows final kinematics that served then as input for our dynamical code.



%In order to find VIRUS-W's individual fiber velocities and their corresponding errors we first use Cappellari \& Emsellem's (2004) penalized pixel fitting method (PPXF) to get an initial guess for the spectrum's redshift. PPXF works by fitting a linear combination of LOSVD convolved templates to the spectra and penalizing solutions that deviate from Gaussian for low S/N levels (I think this was like this...). For this purpose, we used a library of stellar templates priorly extracted with VIRUS-W using the procedure described in the above subsection. This templates had been redshifted to restframe using their own sky spectra as reference.

%As part of the byproducts of PPXF's fitting, one can get the best fitting linear combination of templates and the residuals from the fit. In order to measure the robustness of the linear combination of templates (i.e., to estimate the error), we characterize the residuals as a normal distribution and resample noise from such distribution. We then proceed to add this noise to the combined templates and create for each original spectrum 1000 artificial spectra with similar S/N. 
%Finally, we do a normalized cross-correlation of the set of artificially created spectra with the best matching template and get the mean and dispersion of this measurements. This two values serve as the radial velocity and error for each of our fiber spectra.

%In addition to our measurements, we also use a set of 328 published velocity measurements for Leo I (Mateo et al. 2008). 35 of those measurements fall within the angular coverage of our data set. Our new data set adds 477 measurements to those 35. Table one includes information on the two data sets.

\end{document}
